\documentclass[a4paper,11pt]{article}
\usepackage{bits}
\usepackage{blindtext}
\usepackage{amsmath}
\begin{document}
\title{Bits and pieces for \LaTeX}
\author{Sebastian B\"oser}
  \maketitle
This package provides a list of randomly assorted but useful bits and pieces for writting my physics
papers.

\section{Characters, symbols and small stuff}
A small collection of special characters and symbol definitions that work in all
environments\ldots\\

\begin{tabular}{rcl}
  \verb|\un{km}|      & 3.1\un{km}   & units set upright, with small space \\
  \verb|\circnumb{3}| & \circnumb{3} & a number with a circle around its (works
    from 0  to 9) \\
  \verb|\degree|      & \degree      & degree symbol \\
  \verb|\degC|        & \degC        & degree Celsius \\
  \verb|\xfrac{A}{B}|       & \xfrac{A}{B} &  fraction set in slanted mode\\
  \verb|\sfrac{A}{B}|       & \sfrac{A}{B} &  fraction set in slanted mode, {\bf
  inline} sized\\
  \verb|\order{10}|   & \order{10} &  order of $\ldots$ \\
  \verb|\underwiggle{text}| & \underwiggle{text} & wiggly line under the text  (slow!)
\end{tabular}\\

\ldots and some that work only in math mode\\

\begin{tabular}{rcl}

  \verb|$\dif$|       & $\frac{\dif N}{\dif t}$ & full differential set in upright
    mode, no italic\\
  \verb|$\eps$|       & $\eps$       &    I like this epsilon more...\\
  \verb|$\veps$|      & $\veps$      &    ... than this one\\
  \verb|$\const$|     & $\const$     &    frequently used, set in upright mode\\
  \verb|$\laplace$|   & $\laplace$   &    in bold-face\\
  \verb|$\bra$ $\ket$| & $\bra{\Psi}\ket{\Psi}$  & doesn't work with operator
    inbetween   \\
  \verb|$\mean{x}$|   & $\mean{x}$   &  used to describe the average  \\
\end{tabular}

\section{Comments}
For side-margin comments, use the \verb|\com[color]{text}| command. Default
color is {\tt NavyBlue}\com{{\tt \color{BrickRed}BrickRed} also looks good}, but you can use any color specified in {\tt dvipsnames}.
Also, here is some text\com[ForestGreen]{quite a few comments here} with some comment 
next to it. The same works in 
\[
  f(x) = ax^{2\com{... or better third power}}
\]
math mode. Known issue: comments in math mode may clash with other comments.
If multiple users work on the same document, it may be quite useful to define
your abbreviation for your comments, such as your initials. Use the 
\verb|\mycomcolor{SB}{MidnightBlue}| \mycomcolor{SB}{MidnightBlue} to define {\tt
SB} as short-hand for your color. For future comments, you can then just use
\verb|\com[SB]{...}| to generate a comment in your color \com[SB]{a comment in
my color}.

\section{Reactions}
The \verb|\decayto[length]| command provides a {\it ''decay''} arrow. The length of
the arrow is set in units of {\tt em}, with a default length of {\tt 1.5em}.\\

{\bf Examples:}\\
\begin{tabular}{rl}
  \verb|\decayto| & $\decayto$ \\
  \verb|\decayto[3]| & $\decayto[3]$ \\
  \verb|\decayto[6]| & $\decayto[6]$ \\
\end{tabular}\\[2\baselineskip]

To display decay chains, this is best used together with the \verb|alignat|
enviroment.
\begin{verbatim}
\begin{alignat*}{3}
  p + N \to X + \,& \pi^{\pm},K^{\pm}   \\
            & \decayto[2.5] && \mu^{\pm} + \nu_\mu \\
            & && \decayto e^{\pm} + {\nu_e} + \nu_\mu  \\
\end{alignat*}
\end{verbatim}
\begin{alignat*}{3}
  p + N \to X + \,& \pi^{\pm},K^{\pm}   \\
            & \decayto[2.5] && \mu^{\pm} + \nu_\mu \\
            & && \decayto e^{\pm} + {\nu_e} + \nu_\mu  \\
\end{alignat*}

\section{Figures}
Space is sparse and we often want to setup figures in an efficient way. In
particular if we have several figures in a row next to each other, it looks
nicer if the layout is such that they have all the same height. Here are a few
commands that try to make such a figure layout easier.

\subsection{One figure}
For one figure to be laid out over the full widht of the page, use the
\verb|\onefig[pos]{myfig}{caption}| command. The optional {\tt pos} argument
tells \LaTeX{} where to put the figure, for which you can give one or more
preferences. \LaTeX{} tries to meet these in the order you provide them.\\

\begin{tabular}{cl}
  {\tt h} & put figure right here \\
  {\tt t} & put figure at the top of the page \\
  {\tt b} & put figure at the bottom of the page \\
  {\tt p} & put figure on this page \\
  {\tt h!,t!,\ldots} & any of the above with {\tt !}: force figure to be
  here/top/... 
\end{tabular}\\

The default for the \verb|\onefig{}| and all other commands in this section is
\verb|[tbp]|, so that the figure appears preferenially at the top, then at the
bottom, then on the page. The figure width is by default set to \verb|0.9\columnwidth|, i.e. the figure will
span 90\% of the width of the line. If you want to chanage this {\bf for all
figures} you can use
\begin{verbatim}
  \setlength{\figwidth}{<your new figure width>}
\end{verbatim}

{\bf Example:}\\
Here is the output of this command, which shows the figure across the full
column. {\tt h!} assures that the figure appears right here.\\
\verb|\onefig[h!]{horizontalfig}{This figure shows a prominent peak in the data}|

\onefig[h!]{horizontalfig.png}{This figure shows a prominent peak in the data}

\subsection{Two or three figures}
If two or more figures are put next to each other, it looks usually best if both figures
have the same height. This is achieved by the \verb|\twofig| and
\verb|\twofigone| commands. The difference between the two commands is the the
first one allows a separate caption for each figure, while the other ones has
once caption for both figures. The syntax is as above
\begin{verbatim}
  \twofig[pos]{myfig1}{Caption of Fig. 1}{myfig2}{Caption of Fig. 2}
  \twofigone[pos]{myfig1}{myfig2}{The common caption}
\end{verbatim}
and similarly for three figures
\begin{verbatim}
  \threefig[pos]
     {myfig1}{Caption of Fig. 1}
     {myfig2}{Caption of Fig. 2}
     {myfig3}{Caption of Fig. 3}
  \threefigone[pos]{myfig1}{myfig2}{myfig3}{The common caption}
\end{verbatim}
The figures are automatically scaled to have the same height and fill the {\tt
figwidth}. In between the figures there is some space, which can be set with 
\begin{verbatim}
  \setlength{\figsep}{<your new figure seperation>}
\end{verbatim}

{\bf Example:}
\begin{verbatim}
  \twofig[h!]
     {horizontalfig.png}{Left figure}
     {verticalfig.png}{Right figure}
\end{verbatim}
\twofig[h!]
   {horizontalfig.png}{Left figure}
   {verticalfig.png}{Right figure}

For the example with three figures in a row, we set the \verb|\figsep| so that
there is a little more space and the figures become a little larger.
\begin{verbatim}
\setlength{\figsep}{0pt}
\threefigone[ht!]
  {horizontalfig.png}
  {verticalfig.png}
  {horizontalfig.png}
  {Data (left), results (middle) and same data again (right). Note how
   both  figures are automatically scaled to have the same height.}
\setlength{\figsep}{0.05\columnwidth}
\end{verbatim}
\setlength{\figsep}{0em}
\threefigone[ht!]
  {horizontalfig.png}
  {verticalfig.png}
  {horizontalfig.png}
  {Data (left), results (middle) and
same data again (right). Note how both figures are automatically scaled to have the
same height.}
\setlength{\figsep}{0.05\columnwidth}

\subsection{Floating figures}
Sometimes it is also nice to have figures surrounded by text. For that we can
use the \verb|\textfig| command.
\begin{verbatim}
\textfig[pos]{width}{myfig}{caption}
\end{verbatim}
The {\tt width} is given as a fraction of the linewidth. There are also
positioning parameters that can be specified\\

\begin{tabular}{rl}
  {\tt r} & forces the figure to the right in a paragraph. \\
  {\tt l} & forces the figure to the left in a paragraph. \\
  {\tt p} & right in a paragraph if the pagenumber is odd, and to the left if
  even. \\
  {\tt v} & use the package option if defined, else same as {\tt p} (\bf default).
\end{tabular}\\

Note that if the figure colides with the page boundary, it is sometimes not
shown at all -- use with care (or include a pagebreak just before).\\

{\bf Example:}
\begin{verbatim}
\pagebreak
\textfig[r]{.4}{horizontalfig.png}{Figure surrounded by text}
\end{verbatim}

\pagebreak
\textfig[r]{.4}{horizontalfig.png}{Figure surrounded by text}
{\color{gray}
\blindtext}


  
%%Usage:
%% \onefig{ mypic }{ The caption text }
%% \twofig{ mypic1 }{ The caption of fig 1 }{mypic2}{ The caption of fig 2 }
%% \twofigone{ mypic1 }{ mypic2 }{ The caption text }
%% The figure placement can be adjusted putting optinal placement characters in
%% square brackets before the figures:
%% \onefig[h!] { mypic }{ The caption text }
%% Placement characters are
%%  h : put figure right here
%%  t : put figure at the top of the page
%%  b : put figure at the bottom of the page
%%  p : put figure on this page
%%  h!,t!,b!: force figure to be here/top/bottom
%% Placement characters can be combined and are then treated in the order they
%% are given in. The default is [tbp]


\end{document}
